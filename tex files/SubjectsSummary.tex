\documentclass[14pt]{matmex-diploma}
\usepackage{xcolor}
\usepackage{hyperref}	
\usepackage{enumitem}
\setlist{nolistsep}
\definecolor{urlcolor}{HTML}{0e0b61}

\hypersetup{
    colorlinks=true,
    linkcolor=black,
    filecolor=magenta,      
    urlcolor=urlcolor,
    pdftitle={Sharelatex Example},
    bookmarks=true,
    pdfpagemode=FullScreen,
}

\begin{document}

\tableofcontents

\section{Математика} 


    \subsection{Математический анализ}

        \subsubsection*{Предел}
            1
        \subsubsection*{Обозначения O() и o()}
            1
        \subsubsection*{Доказательство и применение асимптотических оценок, при необходимости переформулировка в «терминах эпсилон и дельта»}
            1
        \subsubsection*{Непрерывность}
            1
        \subsubsection*{Производная}
            1
        \subsubsection*{Первообразная}
            1
        \subsubsection*{Дифференциал}
            1
        \subsubsection*{Нахождение экстремума функции от одной и от многих переменных}
            1
        \subsubsection*{Формула Тейлора}
            1
        
    \subsection{Дискретная математика и математическая логика}
    
        \subsubsection*{Отображения и отношения и их свойства}
            1
        \subsubsection*{Транзитивное замыкание отношения}
            1
        \subsubsection*{Эквивалентность}
            1
        \subsubsection*{Отношения порядка}
            1
        \subsubsection*{Логика высказываний}
            1
        \subsubsection*{Кванторы}
            1
        \subsubsection*{Метод математической индукции}
            1
        \subsubsection*{Основные понятия теории графов}
            1
        \subsubsection*{Лемма о рукопожатиях}
            1
        \subsubsection*{Критерий двудольности}
            1
        \subsubsection*{Оценки числа ребер}
            1
        \subsubsection*{Характеризация деревьев}      
            1
        
    \subsection{Алгебра и теория чисел}
        
        \subsubsection*{Группы}
            1
        \subsubsection*{Поля}
            1
        \subsubsection*{Кольца}
            1
        \subsubsection*{Факторизация}
            1
        \subsubsection*{Идеал}
            1
        \subsubsection*{Сравнения}
            1
        \subsubsection*{Алгоритм Евклида}
            1
        \subsubsection*{Теоремы Эйлера и Ферма}
            1
        \subsubsection*{Кольцо многочленов}
            1
        \subsubsection*{Число корней многочлена}
            1
        \subsubsection*{Линейные пространства и операторы}
            1
        \subsubsection*{Базис}
            1
        \subsubsection*{Размерность}
            1
        \subsubsection*{Ранг}
            1
        \subsubsection*{Собственные числа и собственные векторы}
            1
        \subsubsection*{Характеристический многочлен}
            1
    
    \subsection{Теория вероятностей}
    
        \subsubsection*{Зависимые и независимые события}
            1
        \subsubsection*{Условные вероятности}
            1
        \subsubsection*{Формула полной вероятности}
            1
        \subsubsection*{Математическое ожидание}
            1
        \subsubsection*{Второй момент}
            1
        \subsubsection*{Неравенства Маркова и Чебышёва}
            1
            
\section{Алгоритмы и структуры данных}

    Нужно уметь написать код для перечисленных ниже элементарных алгоритмов.

    \subsection{Оценка алгоритмов}
    
        Мы рассчитываем, что вы понимаете, какое количество операций и объём дополнительной памяти необходимы для обсуждаемых алгоритмов и из каких соображений это получается.
    
    \subsection{Простейшие алгоритмы}
    
        \subsubsection*{Поиск заданного элемента}
            Пои
        \subsubsection*{Поиск наибольшего элемента}
            1
        \subsubsection*{Сортировка вставкой}
            1
        \subsubsection*{Сортировка пузырьком}
            1
        \subsubsection*{Быстрая сортировка}
            1
        \subsubsection*{Иерархические сортировки}
            1
    \subsection{Простейшие структуры данных}
    
        \subsubsection*{Массив}
            1
        \subsubsection*{Список}
            1
        \subsubsection*{Стек}
            1
        \subsubsection*{Очередь}
            1
            
\section{Программирование}     

    Нужно знать базовые принципы одного из «традиционных» (C, C++, Java, Python и др.) языков программирования.
    
    \subsubsection*{Основы синтаксиса}
    
    \subsubsection*{Переменные}
    
    \subsubsection*{Условные выражения}
    
    \subsubsection*{Циклы}
    
    \subsubsection*{Массивы}
    
    \subsubsection*{Функции}
    
    \subsubsection*{Рекурсия}
    
    \subsubsection*{Динамическая память}
    
    \subsubsection*{Стек}

\end{document}