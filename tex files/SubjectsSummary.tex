\documentclass[12pt]{matmex-diploma}
\usepackage{xcolor}
\usepackage{hyperref}	
\usepackage{enumitem}
\setlist{nolistsep}
\definecolor{urlcolor}{HTML}{0e0b61}



\hypersetup{
    colorlinks=true,
    linkcolor=black,
    filecolor=magenta,      
    urlcolor=urlcolor,
    pdftitle={Sharelatex Example},
    pdfpagemode=FullScreen,
}

\begin{document}

\tableofcontents

\section{Математика} 


    \subsection{Математический анализ}

        \subsubsection*{Предел}
            1
        \subsubsection*{Обозначения O() и o()}
            1
        \subsubsection*{Доказательство и применение асимптотических оценок, при необходимости переформулировка в «терминах эпсилон и дельта»}
            1
        \subsubsection*{Непрерывность}
            1
        \subsubsection*{Производная}
            1
        \subsubsection*{Первообразная}
            1
        \subsubsection*{Дифференциал}
            1
        \subsubsection*{Нахождение экстремума функции от одной и от многих переменных}
            1
        \subsubsection*{Формула Тейлора}
            1
        
    \subsection{Дискретная математика и математическая логика}
    
        \subsubsection*{Отображения и отношения и их свойства}
            1
        \subsubsection*{Транзитивное замыкание отношения}
            1
        \subsubsection*{Эквивалентность}
            1
        \subsubsection*{Отношения порядка}
            1
        \subsubsection*{Логика высказываний}
            1
        \subsubsection*{Кванторы}
            1
        \subsubsection*{Метод математической индукции}
            1
        \subsubsection*{Основные понятия теории графов}
            1
        \subsubsection*{Лемма о рукопожатиях}
            1
        \subsubsection*{Критерий двудольности}
            1
        \subsubsection*{Оценки числа ребер}
            1
        \subsubsection*{Характеризация деревьев}      
            1
        
    \subsection{Алгебра и теория чисел}
        
        \subsubsection*{Группы}
            Непустое множество G с заданной на нём бинарной операцией $*:\mathrm{G} \times \mathrm{G} \rightarrow \mathrm{G}$ называется группой $(\mathrm {G} ,*)$, если выполнены следующие аксиомы:
            \begin{enumerate}
                \item ассоциативность: \\$\forall (a,b,c\in G)\colon (a*b)*c=a*(b*c)$;
                \item наличие нейтрального элемента: \\$\exists e\in G\quad \forall a\in G\colon (e*a=a*e=a)$;
                \item наличие обратного элемента: \\$\forall a\in G\quad \exists a^{-1}\in G\colon (a*a^{-1}=a^{-1}*a=e)$.
            \end{enumerate}
            
        \subsubsection*{Поля}
            Множество $F$ с введёнными на нём алгебраическими операциями сложения $+$ и умножения $*$ ($+\colon F\times F\to F,\quad *\colon F\times F\to F$, т. е. $\forall a,b\in F\quad (a+b)\in F,\;a*b\in F$) называется полем $\left\langle F,+,*\right\rangle$, если выполнены следующие аксиомы:
            \begin{enumerate}
                \item Коммутативность сложения: \\$\forall a,b\in F\quad a+b=b+a$.
                \item Ассоциативность сложения: \\$\forall a,b,c\in F\quad (a+b)+c=a+(b+c)$.
                \item Существование нулевого элемента: \\$\exists{0}\in F\colon \forall a\in F\quad a+{0}={0}+a=a$.
                \item Существование противоположного элемента: \\$\forall a\in F\;\exists (-a)\in F\colon a+(-a)={0}$.
                \item Коммутативность умножения: \\$\forall a,b\in F\quad a*b=b*a$.
                \item Ассоциативность умножения: \\$\forall a,b,c\in F\quad (a*b)*c=a*(b*c)$.
                \item Существование единичного элемента: \\$\exists e\in F\setminus \{{0}\}\colon \forall a\in F\quad a*e=a$.
                \item Существование обратного элемента для ненулевых элементов: \\$\forall a\in F\colon a\neq {0})\;\exists a^{-1}\in F\colon a*a^{-1}=e$.
                \item Дистрибутивность умножения относительно сложения: \\$\forall a,b,c\in F\quad (a+b)*c=(a*c)+(b*c)$.
            \end{enumerate}            
            
        \subsubsection*{Кольца}
            Множество $R$, на котором заданы две бинарные операции: $+$ и $*$ (называемые сложение и умножение), со следующими свойствами, выполняющимися для любых $a, b, c \in R$:
            \begin{enumerate}
                \item Коммутативность сложения: \\$a+b=b+a$.
                \item Ассоциативность сложения: \\$(a+b)+c=a+(b+c)$.
                \item Существование нулевого элемента: \\$\exists{0}\in R\colon a+{0}={0}+a=a$.
                \item Существование противоположного элемента: \\$\forall a\in R\;\exists (-a)\in R\colon a+(-a)={0}$.
                \item Ассоциативность умножения: \\$(a*b)*c=a*(b*c)$.
                \item Дистрибутивность: \\$\begin{matrix}a* (b+c)=(a*b)+(a*c)\\(b+c)*a=(b*a)+(c*a)\end{matrix}$.
            \end{enumerate}             
            
        \subsubsection*{Факторизация}
            Факторизацией натурального числа называется его разложение в произведение простых множителей. Может быть выполнена, например, \textbf{перебором возможных делителей}. Способ заключается в том, чтобы последовательно делить факторизуемое число $n$ на натуральные числа от $1$ до $\lfloor {\sqrt {n}}\rfloor$. Формально достаточно делить только на простые числа в этом интервале, однако, для этого необходимо знать их множество. На практике составляется таблица простых чисел и производится проверка небольших чисел (например, до $2^{16}$). Для очень больших чисел алгоритм не используется в силу низкой скорости работы.
            
        \subsubsection*{Идеал}
            Для кольца $R$ идеалом называется подкольцо, замкнутое относительно умножения на элементы из $R$.
        
            Идеалом кольца $R$ называется такое подкольцо (подкольцо кольца $(K,+,*)$ рассматривается как подмножество $R\subset K$, замкнутое относительно операций $+$ и $*$ из основного кольца) $I$ кольца $R$, что
            \begin{enumerate}
                \item $\forall i\in I\;\forall r\in R$ произведение $ir\in I$ (условие на правые идеалы);
                \item $\forall i\in I\;\forall r\in R$ произведение $ri\in I$ (условие на левые идеалы);
            \end{enumerate}             
            
        \subsubsection*{Сравнения}
            Если два целых числа $a$ и $b$ при делении на $m$ дают одинаковые остатки, то они называются сравнимыми (или равноостаточными) по модулю числа $m$.
            
            Сравнимость чисел $a$ и $b$ записывается в виде формулы (сравнения): $$a\equiv b{\pmod  {m}}$$. Число $m$ называется модулем сравнения.
            
        \subsubsection*{Алгоритм Евклида}
            Алгоритм Евклида – эффективный алгоритм для нахождения наибольшего общего делителя двух целых чисел.
            
            Пусть $a$ и $b$ — целые числа, не равные одновременно нулю, и последовательность чисел $a>b>r_{1}>r_{2}>r_{3}>r_{4}>\ \dots \ >r_{n}$ определена тем, что каждое $r_{k}$ — это остаток от деления предпредыдущего числа на предыдущее, а предпоследнее делится на последнее нацело, то есть:
            \\$a=bq_{0}+r_{1}$,
            \\$b=r_{1}q_{1}+r_{2}$,
            \\$r_{1}=r_{2}q_{2}+r_{3}$,
            \\$\cdots$ 
            \\$r_{k-2}=r_{k-1}q_{k-1}+r_{k}$,
            \\$\cdots $
            \\$r_{n-2}=r_{n-1}q_{n-1}+r_{n}$,
            \\$r_{n-1}=r_{n}q_{n}$.
            
            Тогда НОД$(a, b)$, наибольший общий делитель $a$ и $b$, равен $r_{n}$, последнему ненулевому члену этой последовательности.
            
        \subsubsection*{Теоремы Эйлера и Ферма}
            \textbf{Теорема Эйлера:} если $a$ и $m$ взаимно просты, то $a^{\varphi(m)} \equiv 1 \pmod m$, где $\varphi(m)$ — функция Эйлера (количество натуральных чисел, меньших $m$ и взаимно простых с ним).
            
            \textbf{Малая теорема Ферма:} если $a$ не делится на простое число $p$, то \\$a^{{p-1}}\equiv 1{\pmod  p}$.
            
        \subsubsection*{Кольцо многочленов}
            \textbf{Многочлен} от $x$ с коэффициентами в поле $k$ — это выражение вида $p=p_{m}x^{m}+p_{m-1}x^{m-1}+\dots +p_{1}x+p_{0}$, где $p_0, \dots, p_m$ — элементы $k$, коэффициенты $p, a, x, x^{2}, \dots$ — формальные символы («степени x»). Такие выражения можно складывать и перемножать по обычным правилам действий с алгебраическими выражениями (коммутативность сложения, дистрибутивность, приведение подобных членов и т. д.). Члены $p_kx^{k}$ с нулевым коэффициентом $p_k$ при записи обычно опускаются. Используя символ суммы, многочлены записывают в более компактном виде: \\$p=p_{m}x^{m}+p_{m-1}x^{m-1}+\dots +p_{1}x+p_{0}=\sum _{k=0}^{m}p_{k}x^{k}$.
            
            Множество всех многочленов с коэффициентами в $k$ образует коммутативное кольцо, обозначаемое $k[x]$ и называемое \textbf{кольцом многочленов} над $k$.
            
        \subsubsection*{Число корней многочлена}
            Корень многочлена (не равного тождественно нулю) $a_{0}+a_{1}x+\dots +a_{n}x^{n}$ над полем K — это элемент $c\in K$ (либо элемент расширения поля K), такой, что выполняются два следующих равносильных условия:
            \begin{itemize}
                \item данный многочлен делится на многочлен $x-c$;
                \item подстановка элемента c вместо x обращает уравнение \\$a_{0}+a_{1}x+\dots +a_{n}x^{n}=0$ в тождество.
            \end{itemize}   
            
            Число корней многочлена степени $n$ не превышает $n$ даже в том случае, если кратные корни учитывать кратное количество раз.

        \subsubsection*{Линейные пространства и операторы}
            \textbf{Линейное пространство} $V\left(F\right)$ над полем $F$ — это упорядоченная четвёрка $(V,F,+,\cdot )$, где
            \begin{itemize}
                \item $V$ — непустое множество элементов произвольной природы, которые называются векторами;
                \item $F$ — поле, элементы которого называются скалярами;
                \item Определена операция сложения векторов $V\times V\to V$, сопоставляющая каждой паре элементов $\mathbf {x} ,\mathbf {y}$ множества $V$ единственный элемент множества $V$, называемый их суммой и обозначаемый $\mathbf {x} +\mathbf {y}$;
                \item Определена операция умножения векторов на скаляры $F\times V\to V$, сопоставляющая каждому элементу $\lambda$  поля $F$ и каждому элементу $\mathbf {x}$ множества $V$ единственный элемент множества $V$, обозначаемый $\lambda \cdot \mathbf {x}$  или $\lambda \mathbf {x}$;
            \end{itemize}              
            причём заданные операции удовлетворяют следующим аксиомам — аксиомам линейного (векторного) пространства:
            \begin{itemize}
                \item $\mathbf {x} +\mathbf {y} =\mathbf {y} +\mathbf {x}$ , для любых $\mathbf {x} ,\mathbf {y} \in V$ (коммутативность сложения);
                \item $\mathbf {x} +(\mathbf {y} +\mathbf {z} )=(\mathbf {x} +\mathbf {y} )+\mathbf {z}$ , для любых $\mathbf {x} ,\mathbf {y} ,\mathbf {z} \in V$ (ассоциативность сложения);
                \item существует такой элемент $\mathbf {0} \in V$, что $\mathbf {x} +\mathbf {0} =\mathbf {0} +\mathbf {x} =\mathbf {x}$ для любого $\mathbf {x} \in V$ (существование нейтрального элемента относительно сложения), называемый нулевым вектором или просто нулём пространства $V$;
                \item для любого $\mathbf {x} \in V$ существует такой элемент $-\mathbf {x} \in V$, что $\mathbf {x} +(-\mathbf {x} )=\mathbf {0}$ , называемый вектором, противоположным вектору $\mathbf {x}$ ;
                \item $\alpha (\beta \mathbf {x} )=(\alpha \beta )\mathbf {x}$  (ассоциативность умножения на скаляр);
                \item $1\cdot \mathbf {x} =\mathbf {x}$  (унитарность: умножение на нейтральный (по умножению) элемент поля $F$ сохраняет вектор).
                \item $(\alpha +\beta )\mathbf {x} =\alpha \mathbf {x} +\beta \mathbf {x}$  (дистрибутивность умножения вектора на скаляр относительно сложения скаляров);
                \item $\alpha (\mathbf {x} +\mathbf {y} )=\alpha \mathbf {x} +\alpha \mathbf {y}$ (дистрибутивность умножения вектора на скаляр относительно сложения векторов).
            \end{itemize}    
            
            \textbf{Линейным отображением (оператором)} векторного пространства $L_{K}$ над полем $K$ в векторное пространство $M_{K}$ над тем же полем $K$ (линейным оператором из $L_{K}$ в $M_{K})$ называется отображение $f\colon L_{K}\to M_{K}$, удовлетворяющее условию линейности:
            \begin{itemize}
                \item $f(x+y)=f(x)+f(y)$,
                \item $f(\alpha x)=\alpha f(x)$.
            \end{itemize}              
            для всех $x,y\in L_{K}$ и $\alpha \in K$.
            
        \subsubsection*{Базис, размерность, ранг}
            \textbf{Рангом} системы строк (столбцов) матрицы $A$ с $m$ строк и $n$ столбцов называется максимальное число линейно независимых строк.
            
            Число столбцов и строк задают \textbf{размерность} матрицы.
            
            Векторы $\mathbf {x} _{1},\mathbf {x} _{2},\dots ,\mathbf {x} _{n}$ называются линейно зависимыми, если существует их нетривиальная линейная комбинация, значение которой равно нулю; то есть $\alpha _{1}\mathbf {x} _{1}+\alpha _{2}\mathbf {x} _{2}+\ldots +\alpha _{n}\mathbf {x} _{n}=\mathbf {0}$ при некоторых коэффициентах $\alpha _{1},\alpha _{2},\ldots ,\alpha _{n}\in F$, причём хотя бы один из коэффициентов $\alpha_i$ отличен от нуля.

            В противном случае эти векторы называются линейно независимыми.

            Число элементов (мощность) максимального линейно независимого множества элементов векторного пространства не зависит от выбора этого множества. Данное число называется \textbf{рангом}, или \textbf{размерностью}, пространства, а само это множество — \textbf{базисом}. Элементы базиса именуют \textbf{базисными векторами}. Размерность пространства чаще всего обозначается символом \textbf{${\rm {dim}}$}.

        \subsubsection*{Собственные числа и собственные векторы}
            Пусть $L$ — линейное пространство над полем $K$, $A\colon L\to L$ — линейное преобразование.

            \textbf{Собственным вектором} линейного преобразования $A$ называется такой ненулевой вектор $x\in L$, что для некоторого $\lambda \in K$ $Ax=\lambda x$.

            \textbf{Собственным значением} (собственным числом) линейного преобразования $A$ называется такое число $\lambda \in K$, для которого существует собственный вектор, то есть уравнение $Ax=\lambda x$ имеет ненулевое решение $x\in L$.

            Упрощённо говоря, собственный вектор — любой ненулевой вектор $x$, который отображается в коллинеарный ему вектор $\lambda x$ оператором $A$, а соответствующий скаляр $\lambda$ называется собственным значением оператора.

        \subsubsection*{Характеристический многочлен}
            Для данной матрицы $A$, $\chi (\lambda )=\det(A-\lambda E)$, где $E$ — единичная матрица, является многочленом от $\lambda$, который называется характеристическим многочленом матрицы $A$.
    
    \subsection{Теория вероятностей}
    
        \subsubsection*{Зависимые и независимые события}
            Два события называются \textbf{независимыми}, если появление одного из них не изменяет вероятность появления другого. Например, если в цехе работают две автоматические линии, по условиям производства не взаимосвязанные, то остановки этих линий являются независимыми событиями.
            
            События называются \textbf{зависимыми}, если одно из них влияет на вероятность появления другого. Например, две производственные установки связаны единым технологическим циклом. Тогда вероятность выхода из строя одной из них зависит от того, в каком состоянии находится другая.


        \subsubsection*{Условные вероятности}
            Вероятность одного события $B$, вычисленная в предположении осуществления другого события $A$, называется условной вероятностью события $B$ и обозначается $P\{B|A\}$.
            
        \subsubsection*{Формула полной вероятности}
            Если событие $A$ наступает только при условии появления одного из событий $B_1,B_2,\ldots{B_n}$, образующих полную группу несовместных событий, то вероятность события $A$ равна сумме произведений вероятностей каждого из событий $B_1,B_2,\ldots{B_n}$ на соответствующую условную вероятность события $B_1,B_2,\ldots{B_n}$: $P\{A\}=\sum\limits_{i=1}^{n}P\{B_i\}P\{A|B_i\}$.

            При этом события $B_i,~i=1,\ldots,n$ называются гипотезами, а вероятности $P\{B_i\}$ — априорными.
            
        \subsubsection*{Математическое ожидание}
            Математическое ожидание дискретной случайной величины $X$ вычисляется как сумма произведений значений $x_i$, которые принимает случайная величина $X$, на соответствующие вероятности $p_i$: $M[X]=\sum \limits _{{i=1}}^{{\infty }}x_{i}\,p_{i}$.
            
            \textbf{Задание}. Вероятность попадания в цель при одном выстреле равна 0,8 и уменьшается с каждым выстрелом на 0,1. Составить закон распределения числа попаданий в цель, если сделано три выстрела. Найти математическое ожидание, этой случайной величины.
            
            \textbf{Решение}. Введем дискретную случайную величину $X$ = (Число попаданий в цель). X может принимать значения 0, 1, 2 и 3. Найдем соответствующие вероятности. Вероятность не попасть 3 раза: $0,2*0,3*0,4$. Вероятность не попасть 2 раза: $0,2*0,3*0,6+0,2*0,7*0,4+0,8*0,3*0,4$. И т.д. Мат. ожидание будет $0*0,2*0,3*0,4 + 1*0,2*0,3*0,6+0,2*0,7*0,4+0,8*0,3*0,4$ и т.д.
            
        \subsubsection*{Второй момент}
            Начальным моментом s-го порядка прерывной случайной величины называется сумма вида: $\alpha_s[X]=\sum \limits _{{i=1}}^{{\infty }}x_{i}^{s}\,p_{i}$.
            
            Математическое ожидание – первый начальный момент случайной величины.
            
        \subsubsection*{Неравенства Маркова и Чебышёва}
            \textbf{Неравенство Маркова} дает вероятностную оценку того, что значение неотрицательной случайной величины превзойдет некоторую константу через известное математическое ожидание. Когда никаких других данных о распределении нет, неравенство дает некоторую информацию, хотя зачастую оценка груба или тривиальна.
            
            Пусть $X$ - случайная величина, принимающая неотрицательные значения, $M(X)$ - ее конечное математическое ожидание, то для любых $a>0$ выполняется: $P(X \geq a) \leq {M(X)\over{a}}$.
            
            \textbf{Задача}: Среднее количество вызовов, поступающих на коммутатор завода в течение часа, равно 300. Оценить вероятность того, что в течение следующего часа число вызовов на коммутатор превысит 400.
            \textbf{Решение}: По условию $M(X) = 300$. Воспользуемся формулой (неравенством Маркова): $P(X \geq 400) \leq {300\over{400}} = 0,75$, т.е. вероятность того, что число вызовов превысит 400, будет не более 0,75.
            
            \textbf{Неравенство Чёбышева} показывает, что случайная величина принимает значения близкие к среднему (математическому ожиданию) и дает оценку вероятности больших отклонений.
            
            $P(|X-M(X)| \geq a) \leq {D(X)\over{a^{2}}}, a>0$
            
\section{Алгоритмы и структуры данных}

    Нужно уметь написать код для перечисленных ниже элементарных алгоритмов.

    \subsection{Оценка алгоритмов}
    
        Мы рассчитываем, что вы понимаете, какое количество операций и объём дополнительной памяти необходимы для обсуждаемых алгоритмов и из каких соображений это получается.
    
    \subsection{Простейшие алгоритмы}
    
        \subsubsection*{Поиск заданного элемента}
            Пои
        \subsubsection*{Поиск наибольшего элемента}
            1
        \subsubsection*{Сортировка вставкой}
            1
        \subsubsection*{Сортировка пузырьком}
            1
        \subsubsection*{Быстрая сортировка}
            1
        \subsubsection*{Иерархические сортировки}
            1
    \subsection{Простейшие структуры данных}
    
        \subsubsection*{Массив}
            1
        \subsubsection*{Список}
            1
        \subsubsection*{Стек}
            1
        \subsubsection*{Очередь}
            1
            
\section{Программирование}     

    Нужно знать базовые принципы одного из «традиционных» (C, C++, Java, Python и др.) языков программирования.
    
    \subsubsection*{Основы синтаксиса}
    
    \subsubsection*{Переменные}
    
    \subsubsection*{Условные выражения}
    
    \subsubsection*{Циклы}
    
    \subsubsection*{Массивы}
    
    \subsubsection*{Функции}
    
    \subsubsection*{Рекурсия}
    
    \subsubsection*{Динамическая память}
    
    \subsubsection*{Стек}

\end{document}